\chapter{Introdução}

Jogos \textit{online multiplayer} são populares e lucrativos, mantendo-se sempre em crescimento
na indústria de jogos. Em 2016, por exemplo, a taxa de crescimento econômico dos jogos \textit{online} aumentou em 4.2\%, com uma taxa de rendimento de 27\% (Figura \ref{fig:games}) do total de rendimento de todos os tipos de jogos. Parte deste crescimento deriva dos jogos MMO's (\textit{Massive Multiplayer Online}), estilo de jogo onde grandes quantidades de jogadores jogam simultaneamente através da \textit{internet}.


\begin{figure}[h!]
	\begin{center}
		\includegraphics[width=0.6\textwidth]{imagens/1.jpg}
		\caption[Figura ilustrando rendimentos do mercado de jogos no ano de 2016.]{Figura ilustrando rendimentos do mercado de jogos no ano de 2016.}
		\fonte{Adaptado de \citeonline{newzoo}.}
		\label{fig:games}	
	\end{center}
\end{figure}

Porém, enquanto crescem e se tornam uma das aplicações mais populares para a Internet, as trapaças se tornaram um fenômeno chamativo no estado atual dos jogos na \textit{internet}, atuando não só negativamente nas experiências dos outros jogadores como também como um problema de segurança para o jogo e a empresa responsável por ele \cite{trends}. 


Desde sua popularização, a indústria de jogos \textit{online} vem sendo alvo de diferentes tipos de trapaças, que foram evoluindo e modificando-se ao longo dos anos. Muitas delas acabaram causando problemas financeiros para a empresa responsável do jogo. \textit{Age of Empires} e \textit{America’s Army} são alguns exemplos de jogos \textit{online} que sofreram
perdas substanciais devido ao uso de trapaças por seus jogadores \cite{cheatingonlinegames}. 
     
Jogos de \textit{Networked Virtual Environments} (NVEs), reconhecidos pela grande imersão que trazem aos jogadores, seja pela qualidade gráfica ou ambientação de qualidade, acabam sofrendo muito com as trapaças. Como a sensibilidade para perceber atividades ilícitas se torna maior pela ambientação grande deste tipo de jogo, o impacto negativo aos jogadores acaba sendo ainda pior. As trapaças consequentemente tornaram-se uma das formas mais rápidas de destruir a comunidade de um jogo e sua reputação comercial.  

Sabendo do risco e dos problemas de segurança, diversos estudos e trabalhos surgiram para mitigar ou detectar diferentes formas de trapaças que se proliferam com a popularização dos jogos \textit{online} no mercado. Entretanto, assim como alguns tipos de trapaças puderam ser inviabilizadas com boas escolhas de \textit{design} e alterações mínimas, outras formas acabaram necessitando de recursos computacionais altos para serem detectados. Relações de perda e ganho podem ser questionadas pelos criadores do \textit{software}, onde se pode trocar uma maior segurança do jogo por um desempenho e responsividade inferior.


Uma exemplo de vulnerabilidade que pode ser explorada pelos trapaceiros, por exemplo, é a adulteração do \textit{software} que o cliente utiliza no serviço de jogo. Em um modelo cliente-servidor, em que o cliente interage com o servidor, solicitando serviços, e o servidor atende e responde as solicitações, se a mensagem for modificada de forma a tornar-se inválida de acordo com os padrões e regras impostas pelo servidor do jogo, o servidor deve perceber sua ilegitimidade. Caso a ação não seja averiguada no servidor como  inválida, a integridade da partida pode ser comprometida. Trapaças como movimentações ilegais, atributos de personagem acima de seu devido valor, ou mesmo alterações em metadados da partida podem ocorrer devido a este tipo de trapaça. Outras trapaças, como a modificação do \textit{driver} gráfico utilizado para remover as texturas de paredes, são exemplos comuns que podem ser encontrados hoje em dia.


Este trabalho tem como objetivo estudar diferentes comportamentos que podem ser adotados para prevenir ou detectar trapaças em jogos \textit{online}. No Capítulo 2, por existir uma grande variedade de trapaças nos jogos \textit{online}, foram estudados e catalogados os tipos de trapaças existentes em diferentes conjuntos. O trabalho foca em um destes conjuntos de trapaças, abordando estratégias para detectar ou dificultar os tipos de trapaça encontrados neste conjunto. No Capítulo 2.2, os diferentes métodos e estratégias são estudados e analisados, enquanto há um foco maior em dois métodos conhecidos na literatura: uso da execução simbólica para encontrar trapaças, e sistema de auditoria para verificar ações inválidas a partir da definição de estados abstratos e concretos.


O Capítulo 3 apresenta a criação do jogo \textit{online Shooterman}, implementado para este trabalho na linguagem C++, onde se descreve a aplicação tanto do cliente quanto do servidor. Juntamente com o jogo, a implementação das duas estratégias sobre \textit{Shooterman} é demonstrada. No capítulo 4, os resultados obtidos dos dois métodos implementados sobre o jogo \textit{Shooterman} são avaliados e comparados.
